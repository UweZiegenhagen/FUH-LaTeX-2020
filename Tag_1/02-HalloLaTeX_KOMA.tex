\documentclass[ngerman,12pt,parskip=half]{scrartcl}

\usepackage{babel}
\usepackage{blindtext}

\author{Uwe Ziegenhagen}
\title{Mein allererstes \LaTeX-Dokument}

\begin{document}
\maketitle

\tableofcontents

\section{Einleitung}\label{sec:einleitung}
\subsection{Literaturüberblick}

\blindtext

\subsubsection{Deutschsprachige Literatur}

\blindtext

\subsubsection{Englischsprachige Literatur}

\blindtext

\paragraph{Nicht abgesetzt} Hallo, ich bin der Uwe aus Köln.

\subparagraph{Nicht abgesetzt} Hallo, ich bin der Uwe aus Köln.

Eigenen Code oder Demo-Code schreiben/kompilieren und auf den ESP hochladen. Bei mir hat es mit den folgenden Einstellungen geklappt, zusätzlich musste ich aber beim Upload (nach dem Drücken des Upload-Knopfs) einmal den Boot-Button auf dem ESP drücken.

Eigenen Code oder Demo-Code schreiben/kompilieren und auf den ESP hochladen. Bei mir hat es mit den folgenden Einstellungen geklappt, zusätzlich musste ich aber beim Upload (nach dem Drücken des Upload-Knopfs) einmal den Boot-Button auf dem ESP drücken.

Eigenen Code oder Demo-Code schreiben/kompilieren und auf den ESP hochladen. Bei mir hat es mit den folgenden Einstellungen geklappt, zusätzlich musste ich aber beim Upload (nach dem Drücken des Upload-Knopfs) einmal den Boot-Button auf dem ESP drücken.

\blindtext[10]

\section{Hauptteil}

\blindtext[10]

\section{Fazit}

\blindtext[10]


Link zur Einleitung \ref{sec:einleitung}


\end{document}
