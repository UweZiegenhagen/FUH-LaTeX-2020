\documentclass[ngerman,12pt,parskip=half]{scrartcl}

\usepackage{babel}
\usepackage{blindtext}
\usepackage{xcolor}
\usepackage{paralist}

\author{Uwe Ziegenhagen}
\title{Mein allererstes \LaTeX-Dokument}

\newcommand{\autor}[1]{\textbf{\textcolor{green}{#1}}}

\begin{document}

Hallo, ich bin ein \textbf{kurzer Text}, mit dem verschiedene \textit{Textauszeichnungen} gezeigt \textbf{\textit{werden }} sollen. 

Ich kann auch \texttt{Typewriter-Text} darstellen, bei dem jedes Zeichen eine einheitliche Breite hat.

\autor{Hermann Hesse} war ein deutscher Autor.

\autor{Marc-Uwe Kling} ist ein moderner Autor.

\begin{itemize}
	\item Hallo
	\item ich bin 
	\item ein Dokument
	\item das in \LaTeX 
	
	\begin{itemize}
		\item Hallo
		\item ich bin 
		\item ein Dokument
		\item das in \LaTeX
		\item geschrieben
		
		\begin{itemize}
			\item Hallo
			\item ich bin 
			\item ein Dokument
			\item das in \LaTeX
			\item geschrieben
			\item wurde
		\end{itemize}
		
		\item wurde
	\end{itemize}
	
	
	\item geschrieben
	\item wurde
\end{itemize}

\begin{enumerate}
	\item Hallo
	\item ich bin 
	\item ein Dokument
	\item das in \LaTeX
	\item geschrieben
	\item wurde
\end{enumerate}

\begin{compactenum}[i]
	\item Hallo
	\item ich bin 
	\item ein Dokument
	\item das in \LaTeX
	\item geschrieben
	\item wurde
\end{compactenum}


\begin{description}
	\item[Apfel] Äpfel sind Früchte
	\item[Birne] Birnen sind auch Früchte
\end{description}

\end{document}
