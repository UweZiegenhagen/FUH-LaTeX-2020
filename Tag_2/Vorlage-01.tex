\documentclass[a4paper, ngerman, 12pt,parskip=half]{scrreprt}
\usepackage{babel} % für Silbentrennung, eingedeutschte Verzeichnisnamen, etc.
\usepackage[utf8]{inputenc} % UTF8 Encoding
\usepackage[T1]{fontenc}  % Lateinisches Alphabet (T2 für Russisch, etc.)
\usepackage{booktabs} % für schöne Tabellen, siehe Termin 1
\usepackage{blindtext}
\usepackage{graphicx}
\usepackage{microtype} % Mikrotypografie, mit optischem Randausgleich
\usepackage{subcaption}

\begin{document}

\begin{titlepage}
{\large\textbf{Fernuni Hagen \\ 12345 Hagen \\ Lehrstuhl für Angew. Wissenschaften}}

\vspace*{4cm}

{\bfseries\huge Python und Perl im Vergleich -- Warum Python immer gewinnt! }

\begin{center}
	\includegraphics[width=0.2\textwidth]{Bilder/Katze1}
\end{center}

\vfill
Uwe Ziegenhagen \\
Matrikelnummer 31415927 \\
Köln, den \today 
\end{titlepage}

\tableofcontents
\listoffigures
\listoftables
	
\chapter{Einführung in das Thema}	
\section{Literaturüberblick}

In diesem Abschnitt wollen wir kurz die aktuelle Literatur zum Thema besprechen.	
	
\blindtext[10]

\begin{figure}
	\centering
	\includegraphics[width=0.75\textwidth]{Bilder/Katze1}
	\caption{Eine Miezekatze}\label{fig:katze1}
\end{figure}

\blindtext[10]

\begin{figure}
	\centering
	\subcaptionbox{Eine Katze \label{fig:cat1}}
	{\includegraphics[width=0.49\textwidth]{Bilder/Katze1}}
	\subcaptionbox{Die selbe Katze \label{fig:cat2}}
	{\includegraphics[width=0.49\textwidth]{Bilder/Katze2}}
	\caption{Zwei Katzenbilder}\label{fig:katzenbilder}
\end{figure}

Abbildung \ref{fig:cat1} auf Seite \pageref{fig:katzenbilder}

Abbildung \ref{fig:cat2} auf Seite \pageref{fig:katzenbilder}

Abbildung \ref{fig:katzenbilder} auf Seite \pageref{fig:katzenbilder}

\blindtext[2]
	
\end{document}

