% https://de.overleaf.com/learn/latex/Beamer_Presentations:_A_Tutorial_for_Beginners_(Part_4)%E2%80%94Overlay_Specifications
\documentclass[ngerman]{beamer}

\usepackage{babel}
\usepackage{booktabs}
\usepackage{siunitx}

\author{Uwe Ziegenhagen}
\title{Meine erste Präsentation}
\institute{Fernuni Hagen, Lehrstuhl Informatik}
\date{09.10.2020}

\usetheme{Hannover}

\begin{document}

\begin{frame}

\maketitle
	
\end{frame}

\begin{frame}
	
\tableofcontents
	
\end{frame}

\section{Einleitung}

\begin{frame}
	\frametitle{Folientitel}
	
	\begin{itemize}
		\item Hallo
		\item Fernuni
		\item Hagen
		\item Ich 
		\item bin 
		\item Uwe
	\end{itemize}

\end{frame}
	
\begin{frame}
	\frametitle{Folientitel}
	
	\begin{enumerate}
		\item Hallo
		\item Fernuni
		\item Hagen
		\item Ich 
		\item bin 
		\item Uwe
	\end{enumerate}
	
\end{frame}

\begin{frame}
	\frametitle{Meine Katze}
	
\begin{figure}[htb]
	\centering
	\includegraphics[width=0.9\textwidth]{./Bilder/miau.jpg}
	\caption{Das ist eine Katze}\label{fig:mieze}
\end{figure}
	
\end{frame}

\subsection{Einleitung}

\begin{frame}
	\frametitle{Meine Tabelle}
	
\begin{table}
	\caption{Die Überschrift der Tabelle}
\begin{tabular}{lll} \toprule[1.5pt]
	0,859042559	&	0,38670736	&	0,546301257	\\
	0,886732678	&	0,996509183	&	0,155350897	\\
	0,51146953	&	0,070896507	&	0,07723401	\\
	0,944581919	&	0,022187306	&	0,231445556	\\
	0,132974647	&	0,790750011	&	0,830813591	\\
	0,999743171	&	0,867881981	&	0,272954061	\\
	0,288274774	&	0,267472803	&	0,520247612	\\
	42,0 & 2,788915 & 52,4546 \\ \bottomrule[1.5pt]
\end{tabular}
\end{table}
	
\end{frame}

\begin{frame}
\frametitle{}

\begin{columns}
\begin{column}{0.48\textwidth}

\begin{figure}
	\centering
	\includegraphics[width=\textwidth]{./Bilder/miau.jpg}
	\caption{Das ist eine Katze}\label{fig:mieze}
\end{figure}

\end{column}
\begin{column}{0.48\textwidth}
\begin{itemize}
	\item Das
	\item ist 
	\item eine
	\item weiße
	\item Mieze-
	\item katze
	\end{itemize}
\end{column}

\end{columns}


\end{frame}


\begin{frame}
	\frametitle{Fazit}
	
\begin{itemize}
	\item $\Rightarrow$ Karthago muss fallen! \pause
	\item $\rightarrow$ Karthago muss fallen!	\pause
	\item $\Leftarrow$ Karthago muss fallen! \pause
	\item $\leftarrow$ Karthago muss fallen!	
\end{itemize}	

\end{frame}



\begin{frame}
	\frametitle{Fazit 2}
	
\begin{itemize}
	\item<1->  $\Rightarrow$ Karthago muss fallen! 
	\item<2->  $\rightarrow$ Karthago muss fallen!	
	\item<3->  $\Leftarrow$ Karthago muss fallen! 
	\item<4->  $\leftarrow$ Karthago muss fallen!	
\end{itemize}	

\end{frame}

\begin{frame}
	\frametitle{Fazit 3}
	
\begin{itemize}
	\item<-2,4>  $\Rightarrow$ Karthago muss fallen! 
	\item<3->  $\rightarrow$ Karthago muss fallen!	
	\item<1->  \textcolor<2>{orange}{$\Leftarrow$ Karthago muss fallen!} 
	\item<4->  $\leftarrow$ Karthago muss fallen!	
\end{itemize}	

\end{frame}

\begin{frame}
	\frametitle{Fazit}
	
	
Ich bin ein Satz	
\begin{equation}
-\frac{p}{2} \pm \sqrt{   \left(\frac{p}{2}\right)^2  - q   }
\end{equation} und werde abgesetzt gesetzt.

Ich bin eine Formel $a^2+b^2=c^2$ in einem Satz mit TeX-Notation.

Ich bin die gleiche Formel mit LaTeX \( a^2 + b^2 = c^2 \) Notation.

\[  a^2 + b^2 = c^2 \]


	
\end{frame}


	
\end{document}
