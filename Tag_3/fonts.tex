\documentclass[12pt,ngerman,parskip=half]{scrartcl}

\usepackage[utf8]{inputenc}
\usepackage[]{babel}
\usepackage[T1]{fontenc}
\usepackage{blindtext}
\usepackage{microtype}

\newcommand{\setzeschrift}[2]{\fontfamily{#1}\selectfont {\section*{#2}} \blindtext[4]\clearpage}

\begin{document}

\setzeschrift{cmr}{Computer Modern}
\setzeschrift{cmss}{Computer Modern Sans Serif}
\setzeschrift{antt}{Antikwa Torunska}
\setzeschrift{bch}{Bitstream Charter}
\setzeschrift{ccr}{Computer Concrete}
\setzeschrift{DejaVuSerif-TLF}{DejaVu}
\setzeschrift{fav}{Arev}
\setzeschrift{fi4}{Inconsolata}
\setzeschrift{fxb}{Linux Biolinum}
\setzeschrift{fxl}{Linux Libertine}
\setzeschrift{fvs}{Bitstream Vera Sans}
\setzeschrift{jkp}{Kepler}
\setzeschrift{pag}{Avant Garde}
\setzeschrift{pbk}{Bookman}
\setzeschrift{pcr}{Courier}
\setzeschrift{phv}{Helvetica}
\setzeschrift{pnc}{New Century Schoolbook}
\setzeschrift{ppl}{Palatino}
\setzeschrift{ptm}{Times}
\setzeschrift{put}{Utopia}
\setzeschrift{pzc}{Zapf Chancery}
\setzeschrift{qzc}{TeXgyre Chorus}
\setzeschrift{uncl}{Uncial}
\setzeschrift{yes}{Electrum}
\setzeschrift{yly}{Libris}
\setzeschrift{yvt}{Venturis}

\end{document}

Quelle: LaTeX Einführung von Herbert Voß