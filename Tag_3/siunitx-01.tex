%!TEX TS-program = Arara
% arara: pdflatex: {shell: yes}
\documentclass[12pt,ngerman,parskip=half]{scrartcl}

\usepackage[utf8]{inputenc}
\usepackage[T1]{fontenc}
\usepackage{babel}
\usepackage{graphicx}
\usepackage{siunitx}
\usepackage{booktabs}

\begin{document}

\begin{table}
\caption{Standardverhalten des  \texttt{S} Spaltentyps}
\centering
\begin{tabular}{S}
\toprule
{Some Values} \\
\midrule
2.3456 \\
34.2345 \\
-6.7835 \\
90.473 \\
5642,5 \\
1.2e3 \\
1e4 \\
\bottomrule
\end{tabular}
\end{table}

\section{Winkel}

\ang{10}

\ang{10.456}

\ang{10,456}


\section{Zahlen}

\num{3,1415926}

\num{3.1415927e23}

\section{Einheiten}

\si{m^2}

\si{kg.m.s^{-1}}

\si{\kilo\gram\metre\per\square\second} \\
\si{\gram\per\cubic\centi\metre} \\
\si{\square\volt\cubic\lumen\per\farad} \\
\si{\metre\squared\per\gray\cubic\lux} \\
\si{\henry\second}

\SI{3.1415927}{m^2}

\numlist{1;2;3;45;6598; 879}

\numrange{1}{10}

\SIlist{1;2;3;45;6598}{m^2}

Die Räume im Haus waren \SIlist{10;20;23;80}{m^2} groß.

\SIrange{1}{10}{m^2}


\end{document}